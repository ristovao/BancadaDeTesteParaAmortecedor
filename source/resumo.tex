\begin{resumo}[Resumo]

Um dos principais mecanismos de controle de estabilidade de um veículo é o conjunto de amortecedores. Com o objetivo de aumentar a performance de um amortecedor é essencial conhecer seu funcionamento, suas principais características, como o coeficiente de amortecimento, e sua resposta em diferentes situações. O principal objetivo desse projeto é construir uma bancada para testar amortecedores de veículos automotivos. A intenção é desenvolver uma solução de baixo custo que, possivelmente, preencha a lacuna existente desse tipo de solução em meio acadêmico. No processo de implementação, decidiu-se utilizar a metodologia bottom-up, logo a construção do protótipo foi dividida em áreas, especificamente quatro: Projeto Mecânico, que contém o desenvolvimento da estrutura e do mecanismo de excitação utilizando Biela Manivela; Projeto de Software, que centra-se no processamento e interfaceamento dos dados; Projeto Eletrônico, que contém o desenvolvimento do sistema de sensoriamento e controle do teste; Projeto de Energia, que centra-se no acionamento do conjunto motor/inversor, controle de velocidades dos testes e pesquisa sobre métodos matemáticos que expressam a relação entre a temperatura do fluido e a temperatura do corpo do amortecedor.


   \vspace{\onelineskip}
 
   \noindent 
   
   \textbf{Palavras-Chave}: Amortecedor. Bancada de Teste de amortecedor. Coeficiente de amortecimento. Biela Manivela.

\end{resumo}
