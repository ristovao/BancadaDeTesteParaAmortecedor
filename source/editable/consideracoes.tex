\chapter[Considerações Finais]{Considerações Finais}
\label{chap:consideracoes}
	
	É possível observar que no processo de desenvolvimento da bancada foi necessária a execução de diversas tarefas no intuito de caracterizar o teste, em termos funcionais e operacionais, para se obter os parâmetros  imprescindíveis à idealização de um modelo de bancada.
	
	Nesse contexto, definiu-se o panorama geral do projeto e em seguida delimitou-se as áreas de atuação. Em alto nível, em termos de implementação, já foram executadas tarefas como: testes de acionamento e controle de velocidades do conjunto motor/inversor, pesquisas sobre modelos matemáticos que propõem uma estimativa do gradiente de temperatura entre o fluido e o corpo do amortecedor, equacionamento e modelagem do sistema de transmissão, equacionamento e modelagem da biela manivela, simulação e análise estrutural, definição do sistema de sensoriamento, simulação da placa de circuito de condicionamento, implementação da arquitetura cliente/servidor, aquisição parcial dos dados dos sensores, implementação das funções para execução do teste de velocidade fixa.
	
	Dessa forma, estão sendo executadas atividades, conforme os cronogramas, para garantir a conclusão das tarefas setorizadas para, em seguida, trabalhar no processo de integração e construção do protótipo final. 


	