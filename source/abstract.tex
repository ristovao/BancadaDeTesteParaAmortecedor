\begin{resumo}[Abstract]
 \begin{otherlanguage*}{english}
 
One of the main mechanisms to stability’s control of a vehicle is the set of shock absorbers. In order to improve the performance of the shock absorber it’s essential to know their operation, their main characteristics, like dumper coefficient, and their response in different situations. The main goal of this project is to build a stand to test the automotive vehicle shock absorbers. The intention is to develop a low cost solution that, possible, fill a gap on such solutions in academy area. In the implementation process, it was decided to use the bottom-up methodology, so the construction of the prototype was divided in areas, four specifically: Mechanic Project, that contains the development of the structure and the excitation mechanism using the slider-crank; Software Project, that focus on the processing and interfacing of data; Electronic Project, that contains the development of the sensing system and test control; Energy Project, which focuses on activation of the motor/inverter drive, speed control of tests and research on mathematical methods expressing the relationship between the fluid temperature and the damper body temperature.


   \vspace{\onelineskip}
 
   \noindent 

\textbf{Key-words}: Shock Absorber. Dyno dumpers. Dumper coefficient. Slider-Crank

\end{otherlanguage*}
\end{resumo}